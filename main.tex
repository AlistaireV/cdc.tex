\documentclass [11pt]{article}
\usepackage{graphicx} % Required for inserting images
\usepackage[left=2cm,right=2cm,top=0.5cm,bottom=2cm]{geometry}
\usepackage[T1]{fontenc}
\usepackage[utf8]{inputenc}
\usepackage[french]{babel}

\title{\textbf{Projet: BEtag}}
\author{Noah Moussaoui, Alistaire Vionne, Kaan Akman}
\date{8 Décembre 2023}

\begin{document}
\maketitle
\begin{figure}[h]
    \centering
    \includegraphics[scale=0.4]{institution_v3.png}
    \label{ucl}
\end{figure}

\section{Introduction}
Nous sommes le groupe P3A situé à Charleroi sur le campus de l’UCLouvain. Nous nous sommes
engagés à travailler sur un projet qui visait, dans un premier temps, à travailler en groupe, renforcer nos
liens et mettre en commun nos connaissances. Nous avons pu prendre connaissance sur l'UI (User Interface), la cryptographie et les appliquer.

\section{Contexte}
Notre groupe a été recrutée au service d’aide à l’enfance de l’état belge afin de pouvoir suivre l’état d’éveil du bébé et de la quantité de lait consommé par celui-ci. Il nous est demandé de créer un outil permettant de gérer ces fonctions pour permettre l’amélioration de la santé du bébé. Cet outil permettra de communiquer l’état d’éveil du bébé d’un be\string:bi à l’autre et d’enregistrer la quantité de lait consommé pendant la journée. Finalement, ces 2 be\string:bi doivent être capable d’être dans un même canal et de garder leurs messages de façon sécurisés.

\newpage
\section{Fonctionnement du BEtag}
Voici le fonctionnement de notre be\string:bi\string:
\\ Tout d’abord, nous allumons les 2 be\string:bi (parent et enfant) et nous établissons la connexion en interagissant avec l’UI en appuyant sur le bouton B. Nous permettons cette connexion entre ces deux be\string:bi via un mot de passe ‘Betag’ en commun qui permettra de garder la connexion entre ces deux-là et empêcher aux autres d’avoir accès sur le réseau radio. Le key qui permet de garder la connexion entre ces 2 be\string:bi est créé de la façon suivante\string:
be\string:bi parent choisit un mot de passe (P) qui sera crypté en Vigenère avec un générateur de nombre aléatoire (A) sous ce format\string: Type | Longueur | Nonce : Message
\\
\\ Ensuite, P sera décrypté par le be\string:bi enfant qui va utiliser une fonction déterministe F pour calculer le challenge F(A). Ce F(A), dans notre code, sera mis en ‘string’ et finalement subira un hashing.
\noindent Ce hashing sera ensuite rapporté au be\string:bi parent, entre-temps le be\string:bi parent aura fait un hashing aussi pour le calcul de F(A). Finalement, ce be\string:bi parent compare le hashing du be\string:bi parent et du be\string:bi bébé. Si ces deux hashing sont identiques, alors elle définit un key commun avec le be\string:bi bébé.
\\
\\
\noindent Après avoir établi la connexion entre les 2 be\string:bi avec un même key, nous allons pouvoir intéragir avec le menu intéractif du be\string:bi parent. Il y a de telles fonctions comme:
\begin{enumerate}
\item Le compteur de biberon
\item L’état d’éveil du nourrisson
\item La température de la chambre
\item Le contrôle de la veilleuse
\end{enumerate}
\noindent En appuyant sur le bouton A, une liste de fonctions peut être sélectionnée avec une certaine manière. Un compteur est initialisé à 0:
\begin{itemize}
    \item[\textbullet] Si l'on appuie sur A, le compteur diminue d'une unité.
    \item[\textbullet] Si l'on appuie sur B, le compteur augmente d'une unité.
    \item[\textbullet] Si l'on appuie sur le logo, on sélectionne la fonction selon le type donné.
    \item[\textbullet] Si l'on secoue le be\string:bi, elle retourne au menu principal (pour éventuellement rétablir la connexion à nouveau).
\end{itemize}

\noindent On a mis en place un dictionnaire dans lequel ou pourra ajouter le type en string '00' lorsque l'on choisit et si le type est déjà dans le dictionnaire alors il affichera un message d'erreur. Le compteur définit le type de la fonction\string:
\begin{itemize}
    \item[\textbullet] Type == 00 : Établir la connexion
    \item[\textbullet] Type == 01 : Compteur de lait
    \item[\textbullet] Type == 02 : Température mesurée
    \item[\textbullet] Type == 03 : État d'éveil
    \item[\textbullet] Type == 04 : Contrôle de la veilleuse
\end{itemize}
\newpage
\noindent Fonctionnement plus détaillé des fonctions:
\begin{enumerate}
    \item \textbf{Compteur de lait\string:} Sur le be\string:bi parent, le bouton A permet de retirer une unité tandis que le bouton B permet d'ajouter une unité au nombre de biberon. Si on secoue le be\string:bi alors il se réinitialise à 0 sinon en appuyant sur le logo, il retourne au menu.
    \item \textbf{Température mesurée\string:} Sur le be\string:bi enfant, c'est le type 01 qui doit être sélectionné en appuyant le logo. Il envoie le message 'Need heat' au be\string:bi parent si la température <= 19°C. 
    \\Si la température >= 25°C alors il envoie le message 'Need cold'. Sinon, il ne renvoie pas de message.
    \item \textbf{État d'éveil\string:} Sur le be\string:bi enfant, c'est le type 00 et on le sélectionne en appuyant sur le logo. Une première orientation se fait puis une deuxième après 15s. Ensuite, la valeur absolue de ces deux valeurs sont soustraites pour avoir un écart (gap). Si gap <= 15° alors le bébé est endormi. 
    \\Si 15° < gap <= 45° alors le bébé est agité sinon il est très agité. En cas de danger où le bébé se trouve à 180°, un messager d'urgence est envoyé pour une intervention immédiate.
    \item \textbf{Contrôle de la veilleuse\string:} Sur le be\string:bi enfant, c'est le type 02 qui doit être sélectionné en appuyant sur le logo. Si light <= 50 alors il faut activer la veilleuse, si 50 < light <= 90 alors la veilleuse est en mode repos sinon il faut éteindre la veilleuse.
\end{enumerate}
\section{Conclusion}
Pour conclure, notre projet BEtag a connu un grand succès pour ses fonctions de base qui sont l'établissement de connexion entre 2 be\string:bi, le compteur de lait, l'état d'éveil et la détection de mouvements du nourrisson. Nous avons aussi implémenter la fonction de la veilleuse en fonction de la luminosité et la température de la chambre du bébé. Finalement, ce flux de données est crypté et sécurisé.

\end{document}
